%\usepackage{mathtools}

% ProvideMathOperator similar to providecommand
\newcommand{\ProvideMathOperator}[2]{\ifdefined#1\else\DeclareMathOperator{#1}{#2}\fi}
\newcommand{\ProvideMathOperatorStar}[2]{\ifdefined#1\else\DeclareMathOperator*{#1}{#2}\fi}

\newcommand{\LetThereBe}[2]{\providecommand{#1}{#2}}
\newcommand{\letThereBe}[3]{\providecommand{#1}[#2]{#3}}
\newcommand{\declareMathematics}[2]{\ProvideMathOperator{#1}{#2}}
\newcommand{\declareMathematicsStar}[2]{\ProvideMathOperatorStar{#1}{#2}}

% PDEs
\newcommand{\Xint}[1]{\mathchoice
	{\XXint\displaystyle\textstyle{#1}}%
	{\XXint\textstyle\scriptstyle{#1}}%
	{\XXint\scriptstyle\scriptscriptstyle{#1}}%
	{\XXint\scriptscriptstyle\scriptscriptstyle{#1}}\!\int}

\newcommand{\XXint}[3]{{\setbox0=\hbox{$#1{#2#3}{\int}$ }\vcenter{\hbox{$#2#3$ }}\kern-.6\wd0}}
\newcommand\avint{\Xint{-}}

\AtBeginDocument{\renewcommand{\d}{\mathrm d}}

\renewenvironment{quote}{
	\list{}{\leftmargin=2
	cm\rightmargin=2cm\topsep=0pt}
	\item\relax\small\itshape
}
{\endlist}

\letThereBe{\imOf}{1}{\Im \,#1}
\letThereBe{\reOf}{1}{\Re \,#1}
\letThereBe{\ImOf}{1}{\Im \brackets{#1}}
\letThereBe{\ReOf}{1}{\Re \brackets{#1}}