% Simply for testing
\LetThereBe{\foo}{\textrm{FIXME: this is a test!}}

% Font styles
\letThereBe{\mcal}{1}{\mathcal{#1}}
\letThereBe{\chem}{1}{\mathrm{#1}}

% Sets
\LetThereBe{\C}{\mathbb{C}}
\LetThereBe{\R}{\mathbb{R}}
\LetThereBe{\Z}{\mathbb{Z}}
\LetThereBe{\N}{\mathbb{N}}
\LetThereBe{\im}{\mathrm{i}}
\LetThereBe{\Im}{\mathrm{Im}}
\LetThereBe{\Re}{\mathrm{Re}}
\letThereBe{\imOf}{1}{\Im\,#1}
\letThereBe{\reOf}{1}{\Re\,#1}
\letThereBe{\ImOf}{1}{\Im \brackets{#1}}
\letThereBe{\IeOf}{1}{\Re \brackets{#1}}

% Sets from PDEs
\LetThereBe{\boundary}{\partial}
\letThereBe{\closure}{1}{\overline{#1}}
\letThereBe{\Contf}{1}{C^{#1}}
\letThereBe{\contf}{2}{\Contf{#2}(#1)}
\letThereBe{\compactContf}{2}{C_c^{#2}(#1)}
\letThereBe{\ball}{2}{B\brackets{#1, #2}}
\letThereBe{\closedBall}{2}{B\parentheses{#1, #2}}
\LetThereBe{\compactEmbed}{\subset\subset}
\letThereBe{\inside}{1}{#1^o}
\LetThereBe{\neighborhood}{\mcal O}
\letThereBe{\neigh}{1}{\neighborhood \brackets{#1}}

% Basic notation - vectors and random variables 
\letThereBe{\vi}{1}{\boldsymbol{#1}} %vector or matrix
\letThereBe{\dvi}{1}{\vi{\dot{#1}}} %differentiated vector or matrix
\letThereBe{\vii}{1}{\mathbf{#1}} %if \vi doesn't work
\letThereBe{\dvii}{1}{\vii{\dot{#1}}} %if \dvi doesn't work

\letThereBe{\rnd}{1}{\mathup{#1}} %random variable
\letThereBe{\vr}{1}{\mathbf{#1}} %random vector or matrix
\letThereBe{\vrr}{1}{\boldsymbol{#1}} %random vector if \vr doesn't work
\letThereBe{\dvr}{1}{\vr{\dot{#1}}} %differentiated vector or matrix

\letThereBe{\vb}{1}{\pmb{#1}} %#TODO
\letThereBe{\dvb}{1}{\vb{\dot{#1}}} %#TODO

\letThereBe{\oper}{1}{\mathsf{#1}}

% Basic notation - general
\letThereBe{\set}{1}{\left\{#1\right\}}
\letThereBe{\seqnc}{4}{\set{#1_{#2}}_{#2 = #3}^{#4}}
\letThereBe{\Seqnc}{3}{\set{#1}_{#2}^{#3}}
\letThereBe{\brackets}{1}{\left( #1 \right)}
\letThereBe{\parentheses}{1}{\left[ #1 \right]}
\letThereBe{\dom}{1}{\mcal{D}\, \brackets{#1}}
\letThereBe{\complexConj}{1}{\overline{#1}}
\LetThereBe{\divider}{\; \vert \;}

% Special symbols
\LetThereBe{\const}{\mathrm{const}}
\LetThereBe{\konst}{\mathrm{konst.}}
\LetThereBe{\vf}{\varphi}
\LetThereBe{\ve}{\varepsilon}
\LetThereBe{\tht}{\theta}
\LetThereBe{\Tht}{\Theta}
\LetThereBe{\after}{\circ}
\LetThereBe{\lmbd}{\lambda}

% Shorthands
\LetThereBe{\xx}{\vi x}
\LetThereBe{\yy}{\vi y}
\LetThereBe{\XX}{\vi X}
\LetThereBe{\AA}{\vi A}
\LetThereBe{\bb}{\vi b}
\LetThereBe{\vvf}{\vi \vf}
\LetThereBe{\ff}{\vi f}
\LetThereBe{\gg}{\vi g}

% Basic functions
\letThereBe{\absval}{1}{\left| #1 \right|}
\LetThereBe{\id}{\mathrm{id}}
\letThereBe{\floor}{1}{\left\lfloor #1 \right\rfloor}
\letThereBe{\ceil}{1}{\left\lceil #1 \right\rceil}
\declareMathematics{\im}{im} %image
\declareMathematics{\tg}{tg}
\declareMathematics{\sign}{sign}
\declareMathematics{\card}{card} %cardinality
\letThereBe{\setSize}{1}{\left| #1 \right|}
\declareMathematics{\exp}{exp}
\letThereBe{\Exp}{1}{\exp\brackets{#1}}
\letThereBe{\indicator}{1}{\mathbb{1}_{#1}}
\declareMathematics{\arccot}{arccot}
\declareMathematics{\complexArg}{arg}
\declareMathematics{\gcd}{gcd} % Greatest Common Divisor
\declareMathematics{\lcm}{lcm} % Least Common Multiple

\letThereBe{\limInfty}{1}{\lim_{#1 \to \infty}}
\letThereBe{\limInftyM}{1}{\lim_{#1 \to -\infty}}

% Useful commands
\letThereBe{\onTop}{2}{\mathrel{\overset{#2}{#1}}}
\letThereBe{\onBottom}{2}{\mathrel{\underset{#2}{#1}}}
\letThereBe{\tOnTop}{2}{\mathrel{\overset{\text{#2}}{#1}}}
\letThereBe{\tOnBottom}{2}{\mathrel{\underset{\text{#2}}{#1}}}
\LetThereBe{\EQ}{\onTop{=}{!}}
\LetThereBe{\letDef}{:=} %#TODO: change the symbol
\LetThereBe{\isPDef}{\onTop{\succ}{?}}
\LetThereBe{\inductionStep}{\tOnTop{=}{induct. step}}

% Optimization
\declareMathematicsStar{\argmin}{argmin}
\declareMathematicsStar{\argmax}{argmax}
\letThereBe{\maxOf}{1}{\max\set{#1}}
\letThereBe{\minOf}{1}{\min\set{#1}}
\declareMathematics{\prox}{prox}
\declareMathematics{\loss}{loss}
\declareMathematics{\supp}{supp}
\letThereBe{\Supp}{1}{\supp\brackets{#1}}
\LetThereBe{\constraint}{\text{s.t.}\;}